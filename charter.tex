\documentclass[
11pt, % The default document font size, options: 10pt, 11pt, 12pt
%codirector, % Uncomment to add a codirector to the title page
]{charter} 


% El títulos de la memoria, se usa en la carátula y se puede usar el cualquier lugar del documento con el comando \ttitle
\titulo{Optimización biomecánica del ciclismo asistida por inteligencia artificial
} 

% Nombre del posgrado, se usa en la carátula y se puede usar el cualquier lugar del documento con el comando \degreename
%\posgrado{Carrera de Especialización en Sistemas Embebidos} 
%\posgrado{Carrera de Especialización en Internet de las Cosas} 
\posgrado{Carrera de Especialización en Inteligencia Artificial}
%\posgrado{Maestría en Sistemas Embebidos} 
%\posgrado{Maestría en Internet de las cosas}

% Tu nombre, se puede usar el cualquier lugar del documento con el comando \authorname
% IMPORTANTE: no omitir titulaciones ni tildación en los nombres, también se recomienda escribir los nombres completos (tal cual los tienen en su documento)
\autor{Ing. Rodrigo Iván Goñi}

% El nombre del director y co-director, se puede usar el cualquier lugar del documento con el comando \supname y \cosupname y \pertesupname y \pertecosupname
\director{MSc. Fernado Corteggiano}
\pertenenciaDirector{FNRC} 
\codirector{} % para que aparezca en la portada se debe descomentar la opción codirector en los parámetros de documentclass
\pertenenciaCoDirector{FIUBA}

% Nombre del cliente, quien va a aprobar los resultados del proyecto, se puede usar con el comando \clientename y \empclientename
\cliente{Nombre del cliente}
\empresaCliente{Empresa del cliente}
 
\fechaINICIO{24 de junio de 2025}		%Fecha de inicio de la cursada de GdP \fechaInicioName
\fechaFINALPlan{16 de agosto de 2025} 	%Fecha de final de cursada de GdP
\fechaFINALTrabajo{15 de mayo de 2026}	%Fecha de defensa pública del trabajo final


\begin{document}

\maketitle
\thispagestyle{empty}
\pagebreak


\thispagestyle{empty}
{\setlength{\parskip}{0pt}
\tableofcontents{}
}
\pagebreak


\section*{Registros de cambios}
\label{sec:registro}


\begin{table}[ht]
\label{tab:registro}
\centering
\begin{tabularx}{\linewidth}{@{}|c|X|c|@{}}
\hline
\rowcolor[HTML]{C0C0C0} 
Revisión & \multicolumn{1}{c|}{\cellcolor[HTML]{C0C0C0}Detalles de los cambios realizados} & Fecha      \\ \hline
0      & Creación del documento                                 &\fechaInicioName \\ \hline
1      & Se completa hasta el punto 5 inclusive                & 07 de julio de 2025 \\ \hline
2      & Se completa hasta el punto 9 inclusive					 & 15 de julio de 2025 \\ \hline
%		  Se puede agregar algo más \newline
%		  En distintas líneas \newline
%		  Así                                                    & {día} de {mes} de 202X \\ \hline
%3      & Se completa hasta el punto 12 inclusive                & {día} de {mes} de 202X \\ \hline
%4      & Se completa el plan	                                 & {día} de {mes} de 202X \\ \hline

% Si hay más correcciones pasada la versión 4 también se deben especificar acá

\end{tabularx}
\end{table}

\pagebreak



\section*{Acta de constitución del proyecto}
\label{sec:acta}

\begin{flushright}
Buenos Aires, \fechaInicioName
\end{flushright}

\vspace{2cm}

Por medio de la presente se acuerda con el \authorname\hspace{1px} que su Trabajo Final de la \degreename\hspace{1px} se titulará ``\ttitle'' y consistirá en el desarrollo de un prototipo de un sistema inteligente que, mediante la integración de la detección de pose por redes neuronales y el análisis de datos de sensores, optimice los parámetros biomecánicos de la bicicleta para maximizar la potencia, eficiencia y minimizar el riesgo de lesiones del ciclista. El trabajo tendrá un presupuesto preliminar estimado de 600 horas y un costo estimado de \$15000, con fecha de inicio el \fechaInicioName\hspace{1px} y fecha de presentación pública el \fechaFinalName.

Se adjunta a esta acta la planificación inicial.

\vfill

% Esta parte se construye sola con la información que hayan cargado en el preámbulo del documento y no debe modificarla
\begin{table}[ht]
\centering
\begin{tabular}{ccc}
\begin{tabular}[c]{@{}c@{}}Dr. Ing. Ariel Lutenberg \\ Director posgrado FIUBA\end{tabular} & \hspace{2cm} & \begin{tabular}[c]{@{}c@{}}\clientename \\ \empclientename \end{tabular} \vspace{2.5cm} \\ 
\multicolumn{3}{c}{\begin{tabular}[c]{@{}c@{}} \supname \\ Director del Trabajo Final\end{tabular}} \vspace{2.5cm} \\
\end{tabular}
\end{table}




\section{1. Descripción técnica-conceptual del proyecto a realizar}
\label{sec:descripcion}
La biomecánica en el ciclismo es un factor fundamental para mejorar el rendimiento y prevenir lesiones, se basa en el principio de adaptar la bicicleta a las características físicas del ciclista. Un ajuste incorrecto no solo puede causar lesiones, sino también disminuir la potencia de salida hasta en un 20\%. Sin embargo, el acceso a un análisis biomecánico profesional presenta barreras significativas: las soluciones actuales suelen ser costosas, de baja disponibilidad y requieren visitas a laboratorios especializados.

El desafío principal de este proyecto es encontrar el balance óptimo entre la posición que maximiza la velocidad y aquella que minimiza el esfuerzo y el riesgo de lesiones. Frecuentemente, la postura más aerodinámica no es la más sostenible a largo plazo. Para abordar este problema, se propone el desarrollo de un sistema que ajuste automáticamente los parámetros de la bicicleta. Mediante el uso de inteligencia artificial, el sistema analizará la postura del ciclista para optimizar la potencia de salida y reducir la tensión muscular, lo que exige un enfoque de optimización multiobjetivo con diversas restricciones.

Las soluciones convencionales se basan en un análisis estático y puntual, realizado en un único día y dependiente en gran medida de la experiencia del biomecánico. La recomendación de repetir el ajuste anualmente, sumada a su alto costo y escasa disponibilidad, provoca que la mayoría de los ciclistas no mantengan una configuración óptima en sus bicicletas.

El valor fundamental de este sistema radica en ofrecer al ciclista la capacidad de realizar autoajustes frecuentes, de forma autónoma y a un costo significativamente menor que las alternativas tradicionales. Esto democratiza el acceso a una biomecánica de precisión, lo que permite una mejora continua del rendimiento y la prevención de lesiones.

Para lograr estos objetivos, el sistema propuesto se estructura en una serie de módulos interconectados, como se ilustra en el diagrama de bloques de la figura \ref{fig:diagBloques} a continuación:

\begin{figure}[htpb]
\centering
\includegraphics[width=.65\textwidth]{./Figuras/Diagrama_de_Bloques_proyecto_ciclista.pdf}
\caption{Diagrama en bloques del sistema.}
\label{fig:diagBloques}
\end{figure}

\begin{itemize}
\item Entrada de datos: este módulo es el encargado de recolectar datos de las distintas fuentes de información. Se compone de una fuente de video, sensores de rendimiento y los datos antropomórficos del ciclista.
\item Procesamiento y análisis: este módulo toma los datos de entrada y los procesa. Con la ayuda de una red neuronal de detección de pose, añade los datos de posición del ciclista al sistema. Luego, en un módulo de preprocesamiento, los datos se filtran, sincronizan, limpian y completan.
\item Modelado y simulación: con ayuda de un modelo físico y aerodinámico, se predice cómo impactarán los cambios de los parámetros de la bicicleta en el rendimiento.
\item Optimización multi-objetivo: con un algoritmo genético, se buscará maximizar la performance del objetivo en un rango adecuado de posición, tratando de minimizar la resistencia aerodinámica y teniendo en cuenta la morfología y el nivel del ciclista.
\item Salida y retroalimentación: con los datos de la optimización, se generará un reporte de recomendaciones y posibles configuraciones de la bicicleta. Una vez finalizado el reporte, se recomienda ajustar los parámetros para volver a iniciar el análisis.
\end{itemize}




\section{2. Identificación y análisis de los interesados}
\label{sec:interesados}


\begin{table}[ht]
%\caption{Identificación de los interesados}
%\label{tab:interesados}
\begin{tabularx}{\linewidth}{@{}|l|X|X|l|@{}}
\hline
\rowcolor[HTML]{C0C0C0} 
Rol           & Nombre y Apellido & Organización 	& Puesto 	\\ \hline
Auspiciante   &        -           &        -      	&       - 	\\ \hline
Cliente       &   -    &	- &   -     	\\ \hline
Impulsor      &       -            &        -      	&      -  	\\ \hline
Responsable   & \authorname       & FIUBA        	& Alumno 	\\ \hline
Colaboradores &      -             &        -      	&      -  	\\ \hline
Orientador    & \supname	      & \pertesupname 	& Director del Trabajo Final \\ \hline
Equipo        &  		-         &         -     	&     -   	\\ \hline
Opositores    &       -            &        -      	&     -   	\\ \hline
Usuario final &          Ciclistas de distintos tipos         &        -      	&     -   	\\ \hline
\end{tabularx}
\end{table}

\begin{itemize}

\item Responsable: cl \authorname es el líder del proyecto de optimización biomecánica asistida por IA. Es ingeniero en mecatrónica.

\item Orientador: cl \supname	, ingeniero electricista y magister en ciencias de la ingeniería de la UNRC, es director y profesor adjunto. Ya ha dirigido diversas tesis, aportará su experiencia en electrónica, telecomunicaciones y software, para definir el alcance y requerimientos del sistema.
\item Usuario final: ciclistas de distintos tipos que buscan optimizar la comodidad o el rendimiento de su bicicleta a través de ajustes personalizados.
\end{itemize}



\section{3. Propósito del proyecto}
\label{sec:proposito}
El propósito de este proyecto es desarrollar un sistema inteligente y personalizado que, mediante la integración de la detección de pose por redes neuronales y el análisis exhaustivo de datos provenientes de sensores de ciclismo, optimice los parámetros biomecánicos de la bicicleta. Esto incluye la altura y posición del sillín, la longitud de las bielas, la altura y ancho del manillar. El objetivo principal es maximizar la potencia y eficiencia del pedaleo, minimizar el riesgo de lesiones y equilibrar estos factores con la aerodinámica.
Este sistema permitirá a los ciclistas realizar autoajustes frecuentes, de forma autónoma y a un costo significativamente menor que las alternativas tradicionales, lo que democratiza el acceso a una biomecánica de precisión y lo que permite una mejora continua del rendimiento y la prevención de lesiones.

\section{4. Alcance del proyecto}
\label{sec:alcance}

El proyecto incluye:
\begin{itemize}
\item Desarrollo de un sistema de optimización personalizado: proporcionará recomendaciones para el ajuste biomecánico de la bicicleta, que busquen maximizar el rendimiento (medido por la potencia y velocidad), la eficiencia y prevenir lesiones.

\item Ciclo continuo de análisis y retroalimentación: el sistema funcionará a través de las siguientes etapas:
    \begin{itemize}
        \item Captura de datos:
            \begin{itemize}
             	\item Calibración de la cámara.
                \item Grabación de videos del ciclista pedaleando.
                \item Recopilación simultánea de datos de rendimiento mediante sensores de potencia, cadencia y velocidad.
            \end{itemize}
        \item Análisis y modelado:
            \begin{itemize}
                \item Análisis de movimiento: uso de red neuronal de estimación de pose para extraer coordenadas 2D de puntos clave del cuerpo del ciclista desde los videos.
                \item Análisis cinemático: estudio de ángulos de articulaciones, fluidez del pedaleo y variabilidad del movimiento.
                \item Modelo biomecánico: creación de un modelo digital del sistema ciclista y bicicleta para simular el impacto de los ajustes en la potencia y el riesgo de lesión.
            \end{itemize}
        \item Optimización integral: un algoritmo de optimización analizará combinaciones para lograr el equilibrio perfecto entre:
            \begin{itemize}
                \item Ajuste biomecánico: determinación de la configuración óptima de componentes.
                \item Aerodinámica: evaluación de la postura del ciclista para la resistencia del aire, que buscará la posición más aerodinámica y sostenible.
                \item Prevención de lesiones: penalización de configuraciones que aumenten el estrés en articulaciones.
            \end{itemize}
        \item Recomendación y re-evaluación: generación de un reporte con recomendaciones claras para ajustar la bicicleta, lo que permita nuevas sesiones de captura de datos para refinar el ajuste.
    \end{itemize}

\item Adquisición y uso de datos:
    \begin{itemize}
        \item Videos de ciclismo grabados desde el lateral y el frontal.
        \item Datos de sensores sincronizados: potencia, cadencia y velocidad.
        \item Datos antropométricos del ciclista y configuración actual de la bicicleta.
        \item Utilización de recursos propios y entorno controlado para pruebas sistemáticas y sincronización precisa.
    \end{itemize}
\end{itemize}

El presente proyecto no incluye:
\begin{itemize}
\item El desarrollo de hardware personalizado para la captura de datos, más allá de la integración con sensores comerciales existentes.
\item El entrenamiento de la red neuronal de detección de pose desde cero. se espera utilizar o adaptar redes neuronales preexistentes.
\item La integración con todos los posibles modelos de bicicletas y componentes del mercado.
\item La simulación de factores externos complejos como condiciones climáticas extremas o interacciones con el tráfico.
\end{itemize}

\section{5. Supuestos del proyecto}
\label{sec:supuestos}

Para el desarrollo del presente proyecto se supone que:
\begin{itemize}
\item Se dispondrá de un rodillo de entrenamiento inteligente y sus respectivos sensores (potenciómetro, cadencia, velocidad) para la captura de datos en un entorno controlado y la realización de pruebas sistemáticas.
\item Se tendrá acceso a una cámara de video con capacidad para grabar en alta resolución.
\item La red neuronal de detección de pose (como Keypoint R-CNN o MediaPipe) a utilizar será lo suficientemente robusta y precisa para extraer los puntos clave del cuerpo del ciclista necesarios para el análisis biomecánico, y que su rendimiento será adecuado para el procesamiento de video.
\item Existirá suficiente disponibilidad de datos propios y de la comunidad.
\item El entorno de desarrollo y las herramientas de software necesarias son adecuadas.
\item Se contará con el tiempo y los recursos humanos necesarios para la investigación, desarrollo, implementación y realización de pruebas del sistema, incluida la mano de obra propia para la ejecución del proyecto.
\item Se contará con la revisión y retroalimentación constante del director del proyecto, \supname	, para asegurar la correcta orientación técnica y académica.
\item Las condiciones de iluminación durante la captura de video serán adecuadas para permitir una detección de pose precisa.
\item El proyecto se centrará en optimizaciones biomecánicas para el ciclismo en carretera o interior en rodillo.
\item Las variaciones individuales en la anatomía y flexibilidad de los ciclistas podrán ser adecuadamente modeladas y tenidas en cuenta por el algoritmo de optimización.
\end{itemize}

\section{6. Product Backlog}
\label{sec:backlog}

\begin{itemize}
  \item Épica 1: adquisición y preprocesamiento de datos biomecánicos
    \begin{itemize}
      \item HU1: como ciclista, quiero que el sistema tome mi imagen de pedaleo para analizar mi postura.
        \begin{itemize}
          \item Dificultad: 3
          \item Complejidad: 2
          \item Incertidumbre: 2
          \item Suma: 7 $\rightarrow$ Story points: 8
        \end{itemize}
      \item HU2: como ingeniero, quiero consolidar un proceso de obtención de datos coordinado y automatizado para asegurar la eficiencia del análisis.
        \begin{itemize}
          \item Dificultad: 4
          \item Complejidad: 4
          \item Incertidumbre: 3
          \item Suma: 11 $\rightarrow$ Story points: 13
        \end{itemize}
      \item HU3: como ingeniero en inteligencia artificial, quiero aplicar técnicas de visión por computadora para obtener automáticamente los puntos clave de mi cuerpo en los videos de pedaleo, de manera que el análisis biomecánico sea preciso.
        \begin{itemize}
          \item Dificultad: 5
          \item Complejidad: 5
          \item Incertidumbre: 4
          \item Suma: 14 $\rightarrow$ Story points: 21
        \end{itemize}
      \item HU4: como científico de datos, quiero realizar un análisis exploratorio de los datos y filtrarlos correctamente para asegurar la calidad de la información utilizada en los modelos.
        \begin{itemize}
          \item Dificultad: 3
          \item Complejidad: 3
          \item Incertidumbre: 2
          \item Suma: 8 $\rightarrow$ Story points: 8
        \end{itemize}
    \end{itemize}
  \item Épica 2: modelado y optimización biomecánica
    \begin{itemize}
      \item HU5: como ingeniero de software, quiero que el sistema simule las condiciones de pedaleo y el impacto de los ajustes mecánicos para predecir los resultados de la optimización.
        \begin{itemize}
          \item Dificultad: 4
          \item Complejidad: 4
          \item Incertidumbre: 3
          \item Suma: 11 $\rightarrow$ Story points: 13
        \end{itemize}
      \item HU6: como ingeniero, necesito modelar un algoritmo de optimización biomecánica multi-objetivo para encontrar la configuración de bicicleta más eficiente y cómoda.
        \begin{itemize}
          \item Dificultad: 5
          \item Complejidad: 5
          \item Incertidumbre: 5
          \item Suma: 15 $\rightarrow$ Story points: 21
        \end{itemize}
      \item HU7: como sistema de optimización, necesito considerar la morfología y el nivel del ciclista para ofrecer recomendaciones personalizadas y efectivas.
        \begin{itemize}
          \item Dificultad: 4
          \item Complejidad: 3
          \item Incertidumbre: 3
          \item Suma: 10 $\rightarrow$ Story points: 13
        \end{itemize}
    \end{itemize}
  \item Épica 3: generación de recomendaciones y experiencia de usuario
    \begin{itemize}
      \item HU8: como ciclista, quiero recibir un reporte claro y conciso con las recomendaciones de configuración de mi bicicleta para poder realizar los ajustes yo mismo.
        \begin{itemize}
          \item Dificultad: 3
          \item Complejidad: 3
          \item Incertidumbre: 2
          \item Suma: 8 $\rightarrow$ Story points: 8
        \end{itemize}
      \item HU9: como usuario, quiero que la interfaz me permita ingresar fácilmente los parámetros relevantes del ciclista y la bicicleta para obtener un análisis preciso.
        \begin{itemize}
          \item Dificultad: 3
          \item Complejidad: 2
          \item Incertidumbre: 2
          \item Suma: 7 $\rightarrow$ Story points: 8
        \end{itemize}
    \end{itemize}
  \item Épica 4: validación y mejora continua del sistema
    \begin{itemize}
      \item HU10: como ciclista, quiero poder re-evaluar mi postura y rendimiento después de aplicar los ajustes recomendados para refinar la optimización.
        \begin{itemize}
          \item Dificultad: 3
          \item Complejidad: 2
          \item Incertidumbre: 2
          \item Suma: 7 $\rightarrow$ Story points: 8
        \end{itemize}
      \item HU11: como desarrollador, necesito un entorno controlado para realizar pruebas sistemáticas y asegurar la precisión del sistema.
        \begin{itemize}
          \item Dificultad: 4
          \item Complejidad: 3
          \item Incertidumbre: 3
          \item Suma: 10 $\rightarrow$ Story points: 13
        \end{itemize}
    \end{itemize}
\end{itemize}


\section{7. Criterios de aceptación de historias de usuario}
\label{sec:criteriosAceptacion}

\begin{itemize}
  \item Épica 1: adquisición y preprocesamiento de datos biomecánicos
    \begin{itemize}
      \item Criterios de aceptación HU1:
        \begin{itemize}
          \item El sistema debe activar y controlar la cámara de manera autónoma para la captura de video del ciclista mientras pedalea.
          \item El ciclista debe recibir una confirmación visual o sonora clara de que la grabación ha comenzado y finalizado correctamente.
          \item Los videos capturados deben guardarse automáticamente en un formato específico.
        \end{itemize}
      \item Criterios de aceptación HU2:
        \begin{itemize}
          \item Tras la subida de un video, el sistema debe iniciar automáticamente la secuencia de preprocesamiento y extracción de datos.
          \item El ingeniero debe poder visualizar el progreso de la obtención y procesamiento de los datos en una interfaz de estado.
          \item El sistema debe registrar un log detallado de cada paso del proceso de obtención de datos, que incluya la hora de inicio y fin, y cualquier error.
        \end{itemize}
      \item Criterios de aceptación HU3:
        \begin{itemize}
          \item El algoritmo debe identificar correctamente los puntos clave del esqueleto del ciclista en cada fotograma del video.
          \item Los puntos clave detectados deben superponerse visualmente sobre el video original para una verificación rápida de la precisión por parte del ingeniero.
          \item La tasa de detección de puntos clave debe ser adecuada y en un tiempo razonable.
        \end{itemize}
      \item Criterios de aceptación HU4:
        \begin{itemize}
          \item El sistema debe permitir la aplicación de filtros predefinidos a los datos de los puntos clave.
          \item Se deben generar gráficos de distribución y series temporales para cada punto clave y métrica, que permitan la identificación visual de anomalías.
          \item La aplicación de filtros debe reducir el ruido en los datos al menos en un 20\% sin perder información relevante, según métricas preestablecidas.
        \end{itemize}
    \end{itemize}
  \item Épica 2: modelado y optimización biomecánica
    \begin{itemize}
      \item Criterios de aceptación HU5:
        \begin{itemize}
          \item El sistema debe ser capaz de simular las métricas biomecánicas clave para configuraciones de bicicleta diferentes.
          \item Los resultados de la simulación deben presentarse en gráficos comparativos que muestren claramente el impacto de cada ajuste en las métricas.
          \item Cada simulación individual debe completarse en un tiempo adecuado.
        \end{itemize}
      \item Criterios de aceptación HU6:
        \begin{itemize}
          \item El algoritmo debe generar un conjunto de soluciones de Pareto que maximicen la eficiencia, minimicen la incomodidad y respeten las restricciones biomecánicas.
          \item Las soluciones del frente de Pareto deben visualizarse en un gráfico que permita al ingeniero entender el balance entre los diferentes objetivos.
          \item El algoritmo debe demostrar convergencia hacia un conjunto de soluciones estables y diversas dentro de un número razonable de generaciones o iteraciones.
        \end{itemize}
      \item Criterios de aceptación HU7:
        \begin{itemize}
          \item El sistema debe ajustar los rangos de optimización y las ponderaciones de los objetivos con base en los datos de morfología y nivel del ciclista.
          \item Las recomendaciones generadas deben mostrar una justificación clara de cómo la morfología y el nivel del ciclista influyeron en los ajustes sugeridos.
          \item Los modelos internos deben integrar los parámetros morfológicos y de nivel del ciclista como variables de entrada en el proceso de optimización.
        \end{itemize}
    \end{itemize}
  \item Épica 3: generación de recomendaciones y experiencia de usuario
    \begin{itemize}
      \item Criterios de aceptación HU8:
        \begin{itemize}
          \item El sistema debe generar un reporte descargable que incluya las configuraciones de la bicicleta recomendadas.
          \item El reporte debe contener diagramas o imágenes que ilustren visualmente cada ajuste y su ubicación en la bicicleta.
          \item El reporte debe ser compatible con lectores de PDF estándar y poder ser accedido desde dispositivos móviles y de escritorio.
        \end{itemize}
      \item Criterios de aceptación HU9:
        \begin{itemize}
          \item La interfaz debe presentar campos de entrada de datos claros y con etiquetas descriptivas para todos los parámetros requeridos.
          \item La interfaz debe ofrecer validación en tiempo real de los datos ingresados e indicar errores de formato o rangos inválidos de forma intuitiva.
          \item La interfaz debe ser compatible con los navegadores web modernos.
        \end{itemize}
    \end{itemize}
  \item Épica 4: validación y mejora continua del sistema
    \begin{itemize}
      \item Criterios de aceptación HU10:
        \begin{itemize}
          \item El sistema debe permitir al ciclista iniciar un nuevo ciclo de captura de video y análisis de postura para una reevaluación.
          \item El sistema debe generar un informe comparativo que visualice los cambios en la postura y las métricas de rendimiento entre la evaluación inicial y la reevaluación.
          \item Los datos de cada reevaluación deben almacenarse, vincularse con el historial del ciclista y permitir el acceso a versiones anteriores.
        \end{itemize}
      \item Criterios de aceptación HU11:
        \begin{itemize}
          \item El entorno debe permitir la ejecución de pruebas unitarias, de integración y de extremo a extremo para todas las funcionalidades principales del sistema.
          \item Los resultados de las pruebas deben visualizarse en un dashboard o reporte que muestre el estado de las pruebas de forma clara.
          \item El entorno de pruebas debe ser reproducible y configurable para simular diferentes entornos y conjuntos de datos.
        \end{itemize}
    \end{itemize}
\end{itemize}

\section{8. Fases de CRISP-DM}
\label{sec:crisp}

A continuación se detallan las fases del modelo CRISP-DM aplicadas al proyecto.

\begin{enumerate}
  \item Comprensión del negocio:
    \begin{itemize}
      \item Objetivo: el proyecto busca desarrollar un prototipo de sistema inteligente que optimice los parámetros biomecánicos de una bicicleta. El propósito principal es maximizar la potencia y eficiencia del pedaleo, mientras se minimiza el riesgo de lesiones y se equilibra con la aerodinámica para alcanzar la máxima velocidad posible.
      \item Valor agregado de IA: la inteligencia artificial se utilizará para analizar la postura del ciclista a través de redes neuronales de detección de pose y para ejecutar un algoritmo de optimización multiobjetivo.
      \item Métricas de éxito: el éxito del proyecto se medirá por la capacidad del sistema para generar un reporte con recomendaciones claras y cuantificables. Además, la mejora en el rendimiento y la comodidad del ciclista se verificará a través de métricas objetivas y subjetivas post-ajuste, lo que permite un ciclo de reevaluación para validar el impacto positivo de las sugerencias.
    \end{itemize}

  \item Comprensión de los datos:
    \begin{itemize}
      \item Tipo y origen: los datos a utilizar son de diversas fuentes:
        \begin{itemize}
          \item Videos del ciclista: grabaciones desde perspectivas lateral y frontal para el análisis de pose.
          \item Datos de sensores: información de rendimiento como potencia (W), cadencia (RPM) ritmo cardíaco (BPM), y velocidad (km/h), recopilada de forma simultánea a los videos.
          \item Datos antropométricos: medidas del ciclista y de la configuración inicial de su bicicleta.
        \end{itemize}
      \item Cantidad y calidad: se utilizarán recursos propios, que incluyen un rodillo de entrenamiento inteligente que permite realizar pruebas en un entorno controlado. Esta configuración es clave para garantizar la sincronización precisa entre el video y los datos de los sensores. Para enriquecer el dataset, se podrán utilizar datos de plataformas comunitarias como Strava o Zwift.
    \end{itemize}

  \item Preparación de los datos:
    \begin{itemize}
      \item Características clave y transformaciones:
        \begin{itemize}
          \item Se aplicarán técnicas de visión por computadora, mediante una red neuronal de estimación de pose, como Keypoint R-CNN o MediaPipe, para extraer las coordenadas 2D de puntos clave del cuerpo del ciclista a partir de los videos.
          \item Los datos de sensores y los puntos clave extraídos serán filtrados, sincronizados, limpiados y completados en un módulo de preprocesamiento.
          \item A partir de las coordenadas, se realizará un análisis cinemático para estudiar ángulos de articulaciones, fluidez del pedaleo y variabilidad del movimiento.
        \end{itemize}
    \end{itemize}

  \item Modelado:
    \begin{itemize}
      \item Tipo de problema: el núcleo del proyecto es un problema de optimización multiobjetivo. Se busca encontrar el balance óptimo entre la aerodinámica y la potencia, ya que la postura más aerodinámica no siempre es la más potente o sostenible.
      \item Algoritmos posibles: se planea utilizar un algoritmo genético para explorar el espacio de soluciones y encontrar una configuración óptima de la bicicleta. Este algoritmo trabajará sobre un modelo físico que predice cómo los cambios en los parámetros impactan en el rendimiento. Además, se creará un modelo biomecánico digital para simular el efecto de los ajustes.
    \end{itemize}

  \item Evaluación del modelo:
\begin{itemize}
    \item Métricas de rendimiento: la evaluación no se centrará en métricas de clasificación tradicionales. En su lugar, se evaluará la calidad de las soluciones generadas por el algoritmo de optimización. El algoritmo deberá producir un conjunto de soluciones en el frente de Pareto que representen los mejores compromisos posibles entre los objetivos.
    \item Métricas cuantificables y automatizables:
    \begin{itemize}
        \item Reducción de la frecuencia cardíaca (FC) para una potencia dada: a una potencia de salida constante, un menor ritmo cardíaco post-ajuste indicaría una mayor eficiencia cardiovascular y un menor esfuerzo percibido. Esto es directamente medible con los sensores.
        \item Reducción de la variabilidad de ángulos críticos: una menor desviación estándar en ángulos articulares clave a lo largo del ciclo de pedaleo puede indicar mayor fluidez, estabilidad y menor riesgo de lesiones. Esto es automatizable a partir del análisis cinemático.
    \end{itemize}
    \item Métricas subjetivas (para validación funcional):
    \begin{itemize}
        \item Percepción del esfuerzo (RPE): el ciclista reporta su nivel de esfuerzo en una escala de 6 a 20 para una sesión de potencia y duración predefinida. Una disminución del RPE sería un indicador de confort y eficiencia.
        \item Escalas de dolor/molestia: reportes del ciclista sobre la ausencia o reducción de molestias en articulaciones o músculos específicos.
    \end{itemize}
    \item Proceso de evaluación: la validación final será funcional. El sistema generará un reporte con recomendaciones claras. El ciclista aplicará los ajustes y realizará una nueva sesión de captura de datos en el entorno controlado. En esta nueva sesión, se compararán las métricas objetivas y se recopilarán las métricas subjetivas con respecto a la configuración inicial. Esto permitirá cerrar un ciclo de mejora continua y validar empíricamente la efectividad de las recomendaciones.
\end{itemize}
\end{enumerate}

\section{9. Desglose del trabajo en tareas}
\label{sec:wbs}

\begin{longtable}{@{}|l|>{\raggedright\arraybackslash}p{0.5\linewidth}|c|c|@{}}

\caption{Desglose de tareas del proyecto} \label{tab:wbs} \\

\hline
\rowcolor[HTML]{C0C0C0}
Historia de usuario & Tarea técnica & Estimación & Prioridad \\
\hline
\endfirsthead

\multicolumn{4}{c}%
{{\tablename\ \thetable{} -- continuación de la página anterior}} \\
\hline
\rowcolor[HTML]{C0C0C0}
Historia de usuario & Tarea técnica & Estimación & Prioridad \\
\hline
\endhead

\hline \multicolumn{4}{r}{{Continúa en la página siguiente}} \\
\endfoot

\hline
\endlastfoot

% Contenido de la tabla
\multicolumn{4}{|l|}{Épica 1: adquisición y preprocesamiento de datos biomecánicos} \\ \hline
HU1 & Investigar y seleccionar la librería de Python para el control de la cámara. & 4 h & Alta \\ \hline
HU1 & Desarrollar el script para iniciar/detener la grabación y guardar el video. & 8 h & Alta \\ \hline
HU2 & Configurar la recolección de datos de sensores en sincronía con el video. & 8 h & Alta \\ \hline
HU2 & Implementar una función para la carga de datos antropométricos y de la bicicleta. & 6 h & Media \\ \hline
HU2 & Desarrollar el script que automatice la ejecución secuencial de la captura de video y sensores. & 8 h & Alta \\ \hline
HU3 & Investigar y comparar modelos pre-entrenados para la detección de pose. & 8 h & Alta \\ \hline
HU3 & Implementar el modelo seleccionado para procesar los videos y extraer coordenadas 2D de puntos clave. & 8 h & Alta \\ \hline
HU3 & Desarrollar una función para visualizar los puntos clave superpuestos en los fotogramas para verificación. & 6 h & Media \\ \hline
HU4 & Desarrollar scripts para la carga y visualización inicial de datos. & 8 h & Alta \\ \hline
HU4 & Implementar filtros para suavizar el ruido en los datos de sensores y coordenadas. & 8 h & Media \\ \hline
HU4 & Escribir funciones para detectar y gestionar valores atípicos o faltantes en las series de datos. & 6 h & Media \\ \hline
\multicolumn{4}{|l|}{Épica 2: modelado y optimización biomecánica} \\ \hline
HU5 & Desarrollar el modelo físico-matemático que relacione los ángulos articulares con la potencia y la eficiencia. & 8 h & Alta \\ \hline
HU5 & Implementar una función que reciba los parámetros de la bicicleta y simule el impacto en el modelo. & 8 h & Alta \\ \hline
HU5 & Crear visualizaciones para mostrar los resultados de la simulación. & 6 h & Media \\ \hline
HU6 & Investigar y seleccionar una librería de Python para algoritmos genéticos. & 8 h & Alta \\ \hline
HU6 & Definir la función de fitness multiobjetivo. & 8 h & Alta \\ \hline
HU6 & Implementar el algoritmo genético para que explore el espacio de soluciones y encuentre el frente de Pareto. & 8 h & Alta \\ \hline
HU7 & Definir cómo los datos de entrada ajustarán los rangos y pesos del optimizador. & 8 h & Alta \\ \hline
HU7 & Integrar las variables de personalización como parámetros de entrada en el modelo de simulación. & 8 h & Media \\ \hline
HU7 & Ajustar la función de fitness para que penalice soluciones no viables según el perfil del ciclista. & 8 h & Media \\ \hline
\multicolumn{4}{|l|}{Épica 3: generación de recomendaciones y experiencia de usuario} \\ \hline
HU8 & Diseñar la estructura y contenido del reporte final en PDF. & 6 h & Alta \\ \hline
HU8 & Desarrollar el script que genere el reporte en PDF con textos, datos y gráficos de forma automática. & 8 h & Alta \\ \hline
HU9 & Desarrollar una interfaz de usuario simple para la entrada de datos. & 8 h & Media \\ \hline
HU9 & Implementar validaciones de entrada para asegurar que los datos del usuario sean correctos y estén en rango. & 6 h & Baja \\ \hline
\multicolumn{4}{|l|}{Épica 4: validación y mejora continua del sistema} \\ \hline
HU10 & Implementar la funcionalidad para guardar y cargar sesiones de análisis previas. & 8 h & Media \\ \hline
HU10 & Desarrollar un módulo que genere un reporte comparativo entre dos sesiones de un mismo ciclista. & 8 h & Media \\ \hline
HU11 & Estructurar el proyecto para permitir pruebas unitarias de los módulos clave. & 8 h & Alta \\ \hline
HU11 & Crear un conjunto de datos de prueba para validar el pipeline completo. & 8 h & Media \\ \hline
HU11 & Escribir pruebas de integración que verifiquen la correcta comunicación entre los módulos del sistema. & 8 h & Media \\

\end{longtable}

\section{10. Diagrama de Gantt}
\label{sec:gantt}

A continuación se detallan las tareas correspondientes a los identificadores utilizados en el diagrama de Gantt (sección \ref{sec:gantt}).

\begin{longtable}{lp{10cm}l}
\toprule
\textbf{ID} & \textbf{Descripción de la tarea} & \textbf{Tipo} \\
\midrule
\endhead % Encabezado para todas las páginas de la tabla

% --- Gestión y Documentación ---
\multicolumn{3}{l}{\textbf{Grupo G: Gestión y documentación}} \\
G.1 & Planificación y acta constitutiva & No técnica \\
G.2 & Documentación continua & No técnica \\
G.3 & Redacción de memoria & No técnica \\
G.4 & Preparación defensa final & No técnica \\
\midrule

% --- Épica 1 ---
\multicolumn{3}{l}{\textbf{Grupo E1: Adquisición y preprocesamiento de datos}} \\
E1.1 & HU1: Investigar librerías de control de cámara & Técnica \\
E1.2 & HU1: Implementar script de grabación de video & Técnica \\
E1.3 & HU2: Investigar métodos de sincronización de sensores & Técnica \\
E1.4 & HU2: Script para automatización de la captura de datos & Técnica \\
E1.5 & HU3: Investigar modelos de estimación de pose & Técnica \\
E1.6 & HU3: Implementar extracción de keypoints de pose & Técnica \\
E1.7 & HU3: Visualizar esqueleto de puntos clave & Técnica \\
E1.8 & HU4: Scripts de carga de datos y análisis exploratorio (EDA) & Técnica \\
E1.9 & HU4: Implementar filtros para suavizado y reducción de ruido & Técnica \\
\midrule

% --- Épica 2 ---
\multicolumn{3}{l}{\textbf{Grupo E2: Modelado y optimización biomecánica}} \\
E2.1 & HU5: Desarrollar modelo físico-matemático del ciclista & Técnica \\
E2.2 & HU5: Implementar simulación de ajustes posturales & Técnica \\
E2.3 & HU5: Crear visualizaciones del modelo y sus cambios & Técnica \\
E2.4 & HU6: Investigar librerías de optimización genética & Técnica \\
E2.5 & HU6: Definir función de fitness para optimización & Técnica \\
E2.6 & HU6: Implementar algoritmo genético para la optimización & Técnica \\
E2.7 & HU7: Integrar personalización del modelo (datos antropométricos) & Técnica \\
\midrule

% --- Épica 3 ---
\multicolumn{3}{l}{\textbf{Grupo E3: Generación de recomendaciones y UX}} \\
E3.1 & HU9: Desarrollar interfaz de usuario (UI) simple para entrada de datos & Técnica \\
E3.2 & HU9: Implementar validaciones de los datos de entrada & Técnica \\
E3.3 & HU8: Diseñar estructura y plantilla del reporte PDF & Técnica \\
E3.4 & HU8: Script de generación automática de reporte de resultados & Técnica \\
\midrule

% --- Épica 4 ---
\multicolumn{3}{l}{\textbf{Grupo E4: Validación y mejora continua}} \\
E4.1 & HU11: Estructurar el código para pruebas unitarias y de integración & Técnica \\
E4.2 & HU11: Crear dataset de prueba con casos conocidos & Técnica \\
E4.3 & HU11: Escribir pruebas de integración para el flujo completo & Técnica \\
E4.4 & HU10: Implementar guardado y carga de sesiones de análisis & Técnica \\
E4.5 & HU10: Generar reporte comparativo entre diferentes sesiones & Técnica \\

\bottomrule
\caption{Tabla de referencias para las tareas del diagrama de Gantt.}
\label{tab:gantt_ref}
\end{longtable}



\begin{ganttchart}[
    hgrid,
    vgrid,
    x unit=0.64cm, 
    y unit title=0.64cm,
    y unit chart=0.59cm,
    bar/.append style={fill=blue!50!cyan},
    group/.append style={fill=gray!50},
    link/.style={-latex, color=red!80!black},
    bar label node/.append style={align=left},
    group label node/.append style={align=left},
    bar label node/.append style={font=\scriptsize}, 
    group label node/.append style={font=\scriptsize},
    bar height=0.2,
    group height=0.2,
    % Definición de colores sugerida por el usuario
    /pgf/gantt/bar/.append style={fill=techblue},
    /pgf/gantt/group/.append style={fill=groupgray},
    /pgf/gantt/link/.style={-latex, color=linkred}
  ]{1}{25}

\gantttitle{Cronograma del proyecto (semanas)}{25} \\
\gantttitlelist{1,...,25}{1}

% --- GRUPO G: Gestión (No Técnico) ---
\ganttgroup{G}{1}{24} \\
\ganttbar[bar/.append style={fill=nontechgreen}, name=plan]{G.1}{1}{1} \\
\ganttbar[bar/.append style={fill=nontechgreen}, name=docu]{G.2}{2}{23} \\
\ganttbar[bar/.append style={fill=nontechgreen}, name=memoria]{G.3}{20}{23} \\
\ganttbar[bar/.append style={fill=nontechgreen}, name=defensa]{G.4}{24}{24} \\ % Ajustado a 1 semana
\ganttmilestone[name=m1]{28/11/25}{1} \\

% --- GRUPO E1: Épica 1 (Técnico) ---
\ganttgroup{E1}{2}{9} \\
\ganttbar[name=t1_1]{E1.1}{2}{2} \\
\ganttbar[name=t1_2]{E1.2}{3}{4} \\
\ganttbar[name=t1_3]{E1.3}{3}{4} \\
\ganttbar[name=t1_4]{E1.4}{5}{6} \\
\ganttbar[name=t1_5]{E1.5}{4}{5} \\
\ganttbar[name=t1_6]{E1.6}{6}{7} \\
\ganttbar[name=t1_7]{E1.7}{8}{8} \\
\ganttbar[name=t1_8]{E1.8}{7}{8} \\
\ganttbar[name=t1_9]{E1.9}{9}{9} \\
\ganttmilestone[name=m2]{Prep. final.}{9} \\

% --- GRUPO E2: Épica 2 (Técnico) ---
\ganttgroup{E2}{10}{18} \\
\ganttbar[name=t2_1]{E2.1}{10}{11} \\
\ganttbar[name=t2_2]{E2.2}{12}{13} \\
\ganttbar[name=t2_3]{E2.3}{14}{14} \\
\ganttbar[name=t2_4]{E2.4}{10}{10} \\
\ganttbar[name=t2_5]{E2.5}{11}{12} \\
\ganttbar[name=t2_6]{E2.6}{13}{15} \\
\ganttbar[name=t2_7]{E2.7}{16}{18} \\
\ganttmilestone[name=m3]{Mod. fin.}{18} \\

% --- GRUPO E3: Épica 3 (Técnico) ---
% CAMBIO: Adelantado para iniciar en paralelo con E1/E2
\ganttgroup{E3}{9}{21} \\ 
\ganttbar[name=t3_1]{E3.1}{9}{11} \\ % CAMBIO: Inicia mucho antes
\ganttbar[name=t3_2]{E3.2}{12}{12} \\
\ganttbar[name=t3_3]{E3.3}{17}{17} \\
\ganttbar[name=t3_4]{E3.4}{19}{21} \\ % CAMBIO: Inicia después del hito del modelo

% --- GRUPO E4: Épica 4 (Técnico) ---
% CAMBIO: Adelantado para iniciar antes
\ganttgroup{E4}{4}{23} \\
\ganttbar[name=t4_1]{E4.1}{4}{6} \\   % CAMBIO: Inicia mucho antes
\ganttbar[name=t4_2]{E4.2}{16}{17} \\
\ganttbar[name=t4_3]{E4.3}{18}{20} \\
\ganttbar[name=t4_4]{E4.4}{19}{20} \\ % CAMBIO: Inicia después del hito del modelo
\ganttbar[name=t4_5]{E4.5}{21}{22} \\
\ganttmilestone[
  milestone/.append style={fill=red!80!black},
  milestone label node/.append style={color=red!80!black}
]{15/05/26}{25} \\

% --- Enlaces de Dependencia (Ajustados) ---
\ganttlink{t1_1}{t1_2}
\ganttlink{t1_1}{t1_3}
\ganttlink{t1_2}{t1_4}
\ganttlink{t1_5}{t1_6}
\ganttlink{t1_6}{t1_7}
\ganttlink{t1_6}{t1_8}
\ganttlink{t1_8}{t1_9}

% Dependencias de hitos
\ganttlink{m2}{t2_1}
\ganttlink{m2}{t2_4}
\ganttlink{m2}{t3_1} % La UI puede empezar con los datos listos
\ganttlink{m3}{t3_4} % El reporte final necesita el modelo
\ganttlink{m3}{t4_3} % Las pruebas de integración necesitan el modelo
\ganttlink{m3}{t4_4} % Guardar sesión necesita el modelo
\ganttlink{m3}{memoria}

% Dependencias internas de cada grupo
\ganttlink{t2_1}{t2_2}
\ganttlink{t2_2}{t2_3}
\ganttlink{t2_4}{t2_5}
\ganttlink{t2_5}{t2_6}
\ganttlink{t2_6}{t2_7}
\ganttlink{t3_1}{t3_2}
\ganttlink{t3_3}{t3_4}
\ganttlink{plan}{t4_1} % La estructura de pruebas debe seguir a la planificación
\ganttlink{t4_1}{t4_2}
\ganttlink{t4_4}{t4_5}
\ganttlink{t4_5}{defensa}
\end{ganttchart}

\section{11. Planificación de Sprints}

\begin{longtable}{|l|l|p{0.4\linewidth}|c|l|c|}

\caption{Planificación detallada de sprints del proyecto}
\label{tab:sprints} \\

% --- ENCABEZADO PARA LA PRIMERA PÁGINA ---
\hline
\rowcolor[HTML]{C0C0C0}
\textbf{Sprint} & \textbf{HU o Fase} & \textbf{Tarea técnica o de gestión} & \textbf{Horas} & \textbf{Responsable} & \textbf{\% comp.} \\
\hline
\endfirsthead

% --- ENCABEZADO PARA LAS PÁGINAS SIGUIENTES ---
\multicolumn{6}{l}{\tablename\ \thetable{} -- continuación de la página anterior} \\
\hline
\rowcolor[HTML]{C0C0C0}
\textbf{sprint} & \textbf{hu o fase} & \textbf{tarea técnica o de gestión} & \textbf{horas} & \textbf{responsable} & \textbf{\% comp.} \\
\hline
\endhead

% --- PIE DE PÁGINA PARA TODAS MENOS LA ÚLTIMA ---
\hline
\multicolumn{6}{r}{{continúa en la página siguiente}} \\
\endfoot

% --- PIE DE PÁGINA PARA LA ÚLTIMA PÁGINA ---
\hline
\multicolumn{3}{|r|}{\textbf{total de horas estimadas}} & \textbf{600 h} & & \\
\hline
\endlastfoot

% --- CONTENIDO DE LA TABLA ---

Sprint 0 & Planificación & Definición del alcance, cronograma y acta constitutiva. & 15 h & Alumno & 70\% \\
Sprint 0 & Planificación & Configuración del entorno de desarrollo y repositorios. & 10 h & Alumno & 80\% \\ \hline

Sprint 1 & HU1 & Investigación y selección de librería para control de cámara. & 8 h & Alumno & 50\% \\
Sprint 1 & HU1, HU2 & Desarrollo del script de grabación y sincronización de sensores. & 20 h & Alumno & 0\% \\
Sprint 1 & Gestión & Documentación continua del sprint. & 5 h & Alumno & 0\% \\ \hline

Sprint 2 & HU2 & Desarrollo del script para automatizar la captura de datos. & 18 h & Alumno & 0\% \\
Sprint 2 & HU3 & Investigación y comparación de modelos de estimación de pose. & 12 h & Alumno & 100\% \\
Sprint 2 & HU3 & Implementación de la extracción de puntos clave de videos. & 15 h & Alumno & 90\% \\
Sprint 2 & Gestión & Documentación continua del sprint. & 5 h & Alumno & 0\% \\ \hline

Sprint 3 & HU3 & Desarrollo de función para visualización del esqueleto. & 12 h & Alumno & 50\% \\
Sprint 3 & HU4 & Desarrollo de scripts para carga y análisis exploratorio (EDA). & 18 h & Alumno & 0\% \\
Sprint 3 & HU4 & Implementación de filtros para suavizado y limpieza de datos. & 15 h & Alumno & 0\% \\
Sprint 3 & Gestión & Documentación continua del sprint. & 5 h & Alumno & 0\% \\ \hline

Sprint 4 & HU5 & Desarrollo del modelo físico-matemático del ciclista. & 25 h & Alumno & 0\% \\
Sprint 4 & HU5 & Implementación de la simulación de ajustes posturales. & 20 h & Alumno & 0\% \\
Sprint 4 & Gestión & Documentación continua del sprint. & 5 h & Alumno & 0\% \\ \hline

Sprint 5 & HU6 & Investigación y selección de librería para optimización genética. & 10 h & Alumno & 0\% \\
Sprint 5 & HU6 & Definición de la función de fitness multiobjetivo. & 25 h & Alumno & 0\% \\
Sprint 5 & HU9 & Desarrollo de interfaz de usuario para la entrada de datos. & 15 h & Alumno & 0\% \\
Sprint 5 & Gestión & Documentación continua del sprint. & 5 h & Alumno & 0\% \\ \hline

Sprint 6 & HU6 & Implementación del algoritmo genético para la optimización. & 30 h & Alumno & 0\% \\
Sprint 6 & HU7 & Definición de la influencia de datos de entrada en el optimizador. & 15 h & Alumno & 0\% \\
Sprint 6 & Gestión & Documentación continua del sprint. & 5 h & Alumno & 0\% \\ \hline

Sprint 7 & HU7 & Integración de variables de personalización en el modelo. & 20 h & Alumno & 0\% \\
Sprint 7 & HU8 & Diseño de la estructura y plantilla del reporte en PDF. & 10 h & Alumno & 0\% \\
Sprint 7 & HU8 & Desarrollo del script para la generación automática del reporte. & 15 h & Alumno & 0\% \\
Sprint 7 & Gestión & Documentación continua del sprint. & 5 h & Alumno & 0\% \\ \hline

Sprint 8 & HU11 & Creación de un conjunto de datos de prueba con casos conocidos. & 20 h & Alumno & 0\% \\
Sprint 8 & HU11 & Escritura de pruebas de integración para el flujo completo. & 25 h & Alumno & 0\% \\
Sprint 8 & Gestión & Documentación continua del sprint. & 5 h & Alumno & 0\% \\ \hline

Sprint 9 & HU10 & Implementación de guardado y carga de sesiones de análisis. & 20 h & Alumno & 0\% \\
Sprint 9 & HU10 & Desarrollo de reporte comparativo entre sesiones. & 20 h & Alumno & 0\% \\
Sprint 9 & Gestión & Documentación continua del sprint. & 5 h & Alumno & 0\% \\ \hline

Sprint 10 & Memoria & Redacción de secciones iniciales e intermedias de la memoria. & 45 h & Alumno & 0\% \\
Sprint 10 & Gestión & Revisión y ajustes con tutor. & 5 h & Alumno & 0\% \\ \hline

Sprint 11 & Memoria & Redacción final y revisión completa de la memoria. & 25 h & Alumno & 0\% \\
Sprint 11 & Defensa & Preparación de la presentación y material de defensa. & 25 h & Alumno & 0\% \\

\end{longtable}

\section{12. Normativa y cumplimiento de datos (gobernanza)}

En esta sección se debe analizar si los datos utilizados en el proyecto están sujetos a normativas de protección de datos y privacidad, y en qué condiciones se pueden emplear.

\textbf{Aspectos a considerar:}
\begin{itemize}
  \item Evaluar si los datos están regulados por normativas como GDPR, Ley 25.326 de Protección de Datos Personales en Argentina, HIPAA u otras según jurisdicción y temática.
  \item Determinar si el uso de los datos requiere consentimiento explícito de los usuarios involucrados.
  \item Indicar si existen restricciones legales, técnicas o contractuales sobre el uso, compartición o publicación de los datos.
  \item Aclarar si los datos provienen de fuentes licenciadas, de acceso público o bajo algún tipo de autorización especial.
  \item Analizar la viabilidad del proyecto desde el punto de vista legal y ético, considerando la gobernanza de los datos.
\end{itemize}

Este análisis es clave para garantizar el cumplimiento normativo y evitar conflictos legales durante el desarrollo y publicación del proyecto.


\section{13. Gestión de riesgos}
\label{sec:riesgos}

\begin{consigna}{red}
a) Identificación de los riesgos (al menos cinco) y estimación de sus consecuencias:
 
Riesgo 1: detallar el riesgo (riesgo es algo que si ocurre altera los planes previstos de forma negativa)
\begin{itemize}
	\item Severidad (S): mientras más severo, más alto es el número (usar números del 1 al 10).\\
	Justificar el motivo por el cual se asigna determinado número de severidad (S).
	\item Probabilidad de ocurrencia (O): mientras más probable, más alto es el número (usar del 1 al 10).\\
	Justificar el motivo por el cual se asigna determinado número de (O). 
\end{itemize}   

Riesgo 2:
\begin{itemize}
	\item Severidad (S): X.\\
	Justificación...
	\item Ocurrencia (O): Y.\\
	Justificación...
\end{itemize}

Riesgo 3:
\begin{itemize}
	\item Severidad (S):  X.\\
	Justificación...
	\item Ocurrencia (O): Y.\\
	Justificación...
\end{itemize}


b) Tabla de gestión de riesgos:      (El RPN se calcula como RPN=SxO)

\begin{table}[htpb]
\centering
\begin{tabularx}{\linewidth}{@{}|X|c|c|c|c|c|c|@{}}
\hline
\rowcolor[HTML]{C0C0C0} 
Riesgo & S & O & RPN & S* & O* & RPN* \\ \hline
       &   &   &     &    &    &      \\ \hline
       &   &   &     &    &    &      \\ \hline
       &   &   &     &    &    &      \\ \hline
       &   &   &     &    &    &      \\ \hline
       &   &   &     &    &    &      \\ \hline
\end{tabularx}%
\end{table}

Criterio adoptado: 

Se tomarán medidas de mitigación en los riesgos cuyos números de RPN sean mayores a...

Nota: los valores marcados con (*) en la tabla corresponden luego de haber aplicado la mitigación.

c) Plan de mitigación de los riesgos que originalmente excedían el RPN máximo establecido:
 
Riesgo 1: plan de mitigación (si por el RPN fuera necesario elaborar un plan de mitigación).
  Nueva asignación de S y O, con su respectiva justificación:
  \begin{itemize}
	\item Severidad (S*): mientras más severo, más alto es el número (usar números del 1 al 10).
          Justificar el motivo por el cual se asigna determinado número de severidad (S).
	\item Probabilidad de ocurrencia (O*): mientras más probable, más alto es el número (usar del 1 al 10).
          Justificar el motivo por el cual se asigna determinado número de (O).
	\end{itemize}

Riesgo 2: plan de mitigación (si por el RPN fuera necesario elaborar un plan de mitigación).
 
Riesgo 3: plan de mitigación (si por el RPN fuera necesario elaborar un plan de mitigación).

\end{consigna}

\section{14. Sprint Review}
\label{sec:sprint_review}

La revisión de sprint (\emph{Sprint Review}) es una práctica fundamental en metodologías ágiles. Consiste en revisar y evaluar lo que se ha completado al finalizar un sprint. En esta instancia, se presentan los avances y se verifica si las funcionalidades cumplen con los criterios de aceptación establecidos. También se identifican entregables parciales y se consideran ajustes si es necesario.

Aunque el proyecto aún se encuentre en etapa de planificación, esta sección permite proyectar cómo se evaluarán las funcionalidades más importantes del backlog. Esta mirada anticipada favorece la planificación enfocada en valor y permite reflexionar sobre posibles obstáculos.

\textbf{Objetivo:} anticipar cómo se evaluará el avance del proyecto a medida que se desarrollen las funcionalidades, utilizando como base al menos cuatro historias de usuario del \emph{Product Backlog}.


Seleccionar al menos 4 HU del Product Backlog. Para cada una, completar la siguiente tabla de revisión proyectada:

\textbf{Formato sugerido:}
\begin{table}[htpb]
\renewcommand{\arraystretch}{1.5}
\begin{tabular}{|>{\raggedright\arraybackslash}m{2.5cm}|
                >{\raggedright\arraybackslash}m{2.3cm}|
                >{\raggedright\arraybackslash}m{3cm}|
                >{\raggedright\arraybackslash}m{3cm}|
                >{\raggedright\arraybackslash}m{3cm}|}
\hline
\rowcolor[HTML]{CCCCCC}
\textbf{HU seleccionada} & \textbf{Tareas asociadas} & \textbf{Entregable esperado} & \textbf{¿Cómo sabrás que está cumplida?} & \textbf{Observaciones o riesgos} \\
\hline
                         & Tarea 1 &                             &                                           &                                     \\ \cline{2-2}
\multirow{-2}{=}{HU1}    & Tarea 2 & \multirow{-2}{=}{Módulo funcional} & \multirow{-2}{=}{Cumple criterios de aceptación definidos} & \multirow{-2}{=}{Falta validar con el tutor} \\
\hline
                         & Tarea 1 &                             &                                           &                                     \\ \cline{2-2}
\multirow{-2}{=}{HU3}    & Tarea 2 & \multirow{-2}{=}{Reporte generado} & \multirow{-2}{=}{Exportación disponible y clara} & \multirow{-2}{=}{Requiere datos reales} \\
\hline
                         & Tarea 1 &                             &                                           &                                     \\ \cline{2-2}
\multirow{-2}{=}{HU5}    & Tarea 2 & \multirow{-2}{=}{Panel de gestión} & \multirow{-2}{=}{Roles diferenciados operativos} & \multirow{-2}{=}{Riesgo en integración} \\
\hline
                         & Tarea 1 &                             &                                           &                                     \\ \cline{2-2}
\multirow{-2}{=}{HU7}    & Tarea 2 & \multirow{-2}{=}{Informe trimestral} & \multirow{-2}{=}{PDF con gráficos y evolución} & \multirow{-2}{=}{Puede faltar tiempo para ajustes} \\
\hline
\end{tabular}
\end{table}

\section{15. Sprint Retrospective}    
\label{sec:sprint_retro}

La retrospectiva de sprint es una práctica orientada a la mejora continua. Al finalizar un sprint, el equipo (o el alumno, si trabaja de forma individual) reflexiona sobre lo que funcionó bien, lo que puede mejorarse y qué acciones concretas pueden implementarse para trabajar mejor en el futuro.

Durante la cursada se propuso el uso de la \textbf{Estrella de la Retrospectiva}, que organiza la reflexión en torno a cinco ejes:

\begin{itemize}
\item  ¿Qué hacer más?
\item  ¿Qué hacer menos?
\item  ¿Qué mantener?
\item  ¿Qué empezar a hacer?
\item  ¿Qué dejar de hacer?
\end{itemize}

Aun en una etapa temprana, esta herramienta permite que el alumno planifique su forma de trabajar, identifique anticipadamente posibles dificultades y diseñe estrategias de organización personal.

\textbf{Objetivo:} reflexionar sobre las condiciones iniciales del proyecto, identificando fortalezas, posibles dificultades y estrategias de mejora, incluso antes del inicio del desarrollo.


Completar la siguiente tabla tomando como referencia los cinco ejes de la Estrella de la Retrospectiva (\emph{Starfish} o estrella de mar). Esta instancia te ayudará a definir buenas prácticas desde el inicio y prepararte para enfrentar el trabajo de forma organizada y flexible. Se deberá completar la tabla al menos para 3 sprints técnicos y 1 no técnico.

\textbf{Formato sugerido:}

\begin{table}[htpb]
\renewcommand{\arraystretch}{1.4}
\begin{tabular}{|>{\raggedright\arraybackslash}p{1.8cm}|
                >{\raggedright\arraybackslash}p{2.3cm}|
                >{\raggedright\arraybackslash}p{2.3cm}|
                >{\raggedright\arraybackslash}p{2.3cm}|
                >{\raggedright\arraybackslash}p{2.3cm}|
                >{\raggedright\arraybackslash}p{2.3cm}|}
\hline
\rowcolor[HTML]{CCCCCC} 
\textbf{Sprint tipo y N°} & \textbf{¿Qué hacer más?} & \textbf{¿Qué hacer menos?} & \textbf{¿Qué mantener?} & \textbf{¿Qué empezar a hacer?} & \textbf{¿Qué dejar de hacer?} \\
\hline
Sprint técnico - 1 & Validaciones continuas con el alumno & Cambios sin versión registrada & Pruebas con datos simulados & Documentar cambios propuestos & Ajustes sin análisis de impacto \\
\hline
Sprint técnico - 2 & Verificar configuraciones en múltiples escenarios & Modificar parámetros sin guardar historial & Perfiles reutilizables & Usar logs para configuración & Repetir pruebas manuales innecesarias \\
\hline
Sprint técnico - 8 & Comparar correlaciones con casos previos & Cambiar parámetros sin justificar & Revisión cruzada de métricas & Anotar configuraciones usadas & Trabajar sin respaldo de datos \\
\hline
Sprint no técnico - 12 (por ej.: ``Defensa'') & Ensayos orales con feedback & Cambiar contenidos en la memoria & Material visual claro & Dividir la presentación por bloques & Agregar gráficos difíciles de explicar \\
\hline
\end{tabular}
\end{table}




\end{document}